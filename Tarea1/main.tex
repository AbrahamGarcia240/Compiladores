\documentclass[11pt,a4paper]{article}
\usepackage[utf8]{inputenc}
\usepackage[spanish]{babel}
\usepackage{amsmath}
\usepackage{listings}
\usepackage[absolute]{textpos}
\usepackage{amsfonts}
\usepackage{afterpage}
\usepackage{amssymb}
\usepackage{graphicx}
\usepackage[left=2cm,right=2cm,top=2cm,bottom=2cm]{geometry}
\author{Garcia Lomeli Abraham Amos}
\title{Compiladores\\Ejercicios de Análisis Sintáctico Mediante LR0}

\begin{document}

\maketitle
\tableofcontents
\section{Ejercicio 1}
Para este primer ejemplo, se utilizará una gramática que representa de manera parcial a las expresiones artiméticas, la cual se define como:

\begin{itemize}
\item E'$\longrightarrow$ E
\item E $\longrightarrow$ E+T,T
\item T $\longrightarrow$ T*F,F
\item F $\longrightarrow$ (E), num

\end{itemize}

Al numerar cada regla tendremos:

\begin{itemize}
\item \textbf{0} E'$\longrightarrow$ E
\item \textbf{1} E $\longrightarrow$ E+T
\item \textbf{2} E $\longrightarrow$ T
\item \textbf{3} T $\longrightarrow$ T*F
\item \textbf{4} T $\longrightarrow$ F
\item \textbf{5} F $\longrightarrow$ (E)
\item \textbf{6} F $\longrightarrow$ num
\end{itemize}


La cadena a evaluar será:
\begin{center}
\textbf{(num+num)*num*num\$}
\end{center}
\newpage
\subsection{Análisis}
A continuación se muestra el resultado del análisis mediante LR0
\begin{center}

\begin{table}[h]
\begin{tabular}{|l|l|l|}
\hline
\multicolumn{1}{|c|}{\textbf{Pila}} & \multicolumn{1}{c|}{\textbf{Condesa}} & \multicolumn{1}{c|}{\textbf{Acción}} \\ \hline
0 & (num+num)*num*num\$ & D1 \\ \hline
0(1 & num+num)*num*num\$ & D2 \\ \hline
0(1num2 & +num)*num*num\$ & R6 F$\longrightarrow$num \\ \hline
0(1F5 & +num)*num*num\$ & R4 T$\longrightarrow$F \\ \hline
0(1T4 & +num)*num*num\$ & R2 E$\longrightarrow$T \\ \hline
0(1E6 & +num)*num*num\$ & D7 \\ \hline
0(1E6+7 & num)*num*num\$ & D2 \\ \hline
0(1E6+7num2 & )*num*num\$ & R6 F$\longrightarrow$num \\ \hline
0(1E6+7F5 & )*num*num\$ & R4 T$\longrightarrow$F \\ \hline
0(1E6+7T10 & )*num*num\$ & R1 E$\longrightarrow$E+T \\ \hline
0(1E6 & )*num*num\$ & D9 \\ \hline
0(1E6)9 & *num*num\$ & R5 F$\longrightarrow$(E) \\ \hline
0F5 & *num*num\$ & R4 T$\longrightarrow$F \\ \hline
0T4 & *num*num\$ & D8 \\ \hline
0T4*8 & num*num\$ & D2 \\ \hline
0T4*8num2 & *num\$ & R6 F$\longrightarrow$num \\ \hline
0T4*8F11 & *num\$ & R3 T$\longrightarrow$T*F \\ \hline
0T4 & *num\$ & D8 \\ \hline
0T4*8 & num\$ & D2 \\ \hline
0T4*8num2 & \$ & R6 F$\longrightarrow$num \\ \hline
0T4*8F11 & \$ & R3 T$\longrightarrow$T*F \\ \hline
0T4 & \$ & R2 E$\longrightarrow$T \\ \hline
0E3 & \$ & ACCEPT \\ \hline
\end{tabular}
\end{table}


\end{center}

\section{Ejercicio 2}
Se busca obtener la tabla LR0 de la siguiente gramática:

\begin{itemize}
\item E'$\longrightarrow$ E
\item E $\longrightarrow$ wX,aY
\item X $\longrightarrow$ bX,z
\item Y $\longrightarrow$ bY,z

\end{itemize}

\subsection{Solución:}
Primero se realizará la cerradura epsilon:
\begin{itemize}
\item \textbf{C(E'$\longrightarrow$ E)}=\{E'$\longrightarrow$ E, E $\longrightarrow$ •wX,  E $\longrightarrow$ •aY \}=\textit{$S_{0}$} 
\end{itemize}


Luego se procede a analizar $S_{0}$.
\begin{itemize}
\item \textbf{Ir\_A($S_{0}$,E')}=\{E'$\longrightarrow$ E• \}=\textit{$S_{1}$} 
\item \textbf{Ir\_A($S_{0}$,w)}=\{E$\longrightarrow$ w•X, X $\longrightarrow$ •bX, X $\longrightarrow$ •z \}=\textit{$S_{2}$} 
\item \textbf{Ir\_A($S_{0}$,a)}=\{E$\longrightarrow$ a•Y, Y $\longrightarrow$ •bY, Y $\longrightarrow$ •z \}=\textit{$S_{3}$}
\end{itemize}


Como la operaciòn mover de todo conjunto donde • sea el último simbolo es vacia, se omite el analisis de $S_{1}$

Ahora se realizará el análisis para $S_{2}$
\begin{itemize}
\item \textbf{Ir\_A($S_{2}$,X)}=\{E$\longrightarrow$ wX• \}=\textit{$S_{4}$} 
\item \textbf{Ir\_A($S_{2}$,b)}=\{X$\longrightarrow$ b•X \}=\textit{$S_{5}$} 
\item \textbf{Ir\_A($S_{2}$,z)}=\{X $\longrightarrow$ z• \}=\textit{$S_{6}$}
\end{itemize}

Se procede con el análisis de $S_{3}$

\begin{itemize}
\item \textbf{Ir\_A($S_{3}$,Y)}=\{E$\longrightarrow$ aY• \}=\textit{$S_{7}$} 
\item \textbf{Ir\_A($S_{3}$,b)}=\{Y$\longrightarrow$ b•Y \}=\textit{$S_{8}$} 
\item \textbf{Ir\_A($S_{3}$,z)}=\{Y $\longrightarrow$ z• \}=\textit{$S_{9}$}
\end{itemize}

Como la operaciòn mover de todo conjunto donde • sea el último simbolo es vacia, se omite el analisis de $S_{4}$

Se procede con el análisis de $S_{5}$
\begin{itemize}
\item \textbf{Ir\_A($S_{5}$,X)}=\{X$\longrightarrow$ bX• \}=\textit{$S_{10}$} 
\end{itemize}

Como la operaciòn mover de todo conjunto donde • sea el último simbolo es vacia, se omite el analisis de $S_{6}$ y  $S_{7}$


Se procede con el análisis de $S_{8}$
\begin{itemize}
\item \textbf{Ir\_A($S_{8}$,Y)}=\{Y$\longrightarrow$ bY• \}=\textit{$S_{11}$} 
\end{itemize}

Al finalizar lo anterior tendremos en la tabla:

\begin{table}[h]
\begin{tabular}{|l|l|l|l|l|l|l|l|l|}
\hline
\textbf{} & \textbf{w} & \textbf{a} & \textbf{b} & \textbf{z} & \textbf{\$} & \textbf{E} & \textbf{X} & \textbf{Y} \\ \hline
\textbf{0} & d2 & d3 &  &  &  & 1 &  &  \\ \hline
\textbf{1} &  &  &  &  &  &  &  &  \\ \hline
\textbf{2} &  &  & d5 & d6 &  &  & 4 &  \\ \hline
\textbf{3} &  &  & d8 & d9 &  &  &  & 7 \\ \hline
\textbf{4} &  &  &  &  &  &  &  &  \\ \hline
\textbf{5} &  &  &  &  &  &  & 10 &  \\ \hline
\textbf{6} &  &  &  &  &  &  &  &  \\ \hline
\textbf{7} &  &  &  &  &  &  &  &  \\ \hline
\textbf{8} &  &  &  &  &  &  &  & 11 \\ \hline
\textbf{9} &  &  &  &  &  &  &  &  \\ \hline
\textbf{10} &  &  &  &  &  &  &  &  \\ \hline
\textbf{11} &  &  &  &  &  &  &  &  \\ \hline
\end{tabular}
\end{table}

\newpage
Para $S_{1}$, $S_{4}$, $S_{6}$, $S_{7}$, $S_{9}$, $S_{10}$ y $S_{11}$ sucede que el punto está al final, por lo tanto se debe hacer el \textit{follow} de su lado izquierdo, siendo asi se sabe que:

\begin{itemize}
\item Follow(E')=\{\$\}
\item Follow(E)=\{\$\}
\item Follow(X)=\{\$\}
\item Follow(Y)=\{\$\}
\end{itemize}

Por lo tanto la tabla LR0 serà:
\begin{table}[h]
\begin{tabular}{|l|l|l|l|l|l|l|l|l|}
\hline
\textbf{} & \textbf{w} & \textbf{a} & \textbf{b} & \textbf{z} & \textbf{\$} & \textbf{E} & \textbf{X} & \textbf{Y} \\ \hline
\textbf{0} & d2 & d3 &  &  &  & 1 &  &  \\ \hline
\textbf{1} &  &  &  &  & ACCEPT &  &  &  \\ \hline
\textbf{2} &  &  & d5 & d6 &  &  & 4 &  \\ \hline
\textbf{3} &  &  & d8 & d9 &  &  &  & 7 \\ \hline
\textbf{4} &  &  &  &  & ACCEPT &  &  &  \\ \hline
\textbf{5} &  &  &  &  &  &  & 10 &  \\ \hline
\textbf{6} &  &  &  &  & ACCEPT &  &  &  \\ \hline
\textbf{7} &  &  &  &  & ACCEPT &  &  &  \\ \hline
\textbf{8} &  &  &  &  &  &  &  & 11 \\ \hline
\textbf{9} &  &  &  &  & ACCEPT &  &  &  \\ \hline
\textbf{10} &  &  &  &  & ACCEPT &  &  &  \\ \hline
\textbf{11} &  &  &  &  & ACCEPT &  &  &  \\ \hline
\end{tabular}
\end{table}
\end{document}




